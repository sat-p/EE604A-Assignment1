%%% Template originaly created by Karol Kozioł (mail@karol-koziol.net) and modified for ShareLaTeX use

\documentclass[a4paper,fleqn,11pt]{article}

\usepackage[T1]{fontenc}
\usepackage[utf8]{inputenc}
\usepackage{graphicx}
\usepackage{xcolor}

\usepackage{tgtermes}

\usepackage[
pdftitle={EE698G - Probabilistic Mobile Robotics Assignment}, 
pdfauthor={Satya Prakash Panuganti, 14610},
colorlinks=true,linkcolor=blue,urlcolor=blue,citecolor=blue,bookmarks=true,
bookmarksopenlevel=2]{hyperref}
\usepackage{amsmath,amssymb,amsthm,textcomp}
\usepackage{enumerate}
\usepackage{multicol}
\usepackage{tikz}

\usepackage{geometry}
\geometry{total={210mm,297mm},
left=25mm,right=25mm,%
bindingoffset=0mm, top=20mm,bottom=20mm}

\usepackage{ mathrsfs }

\linespread{1.3}

\newcommand{\linia}{\rule{\linewidth}{0.5pt}}

% custom theorems if needed
\newtheoremstyle{mytheor}
    {1ex}{1ex}{\normalfont}{0pt}{\scshape}{.}{1ex}
    {{\thmname{#1 }}{\thmnumber{#2}}{\thmnote{ (#3)}}}

\theoremstyle{mytheor}
\newtheorem{defi}{Definition}

% my own titles
\makeatletter
\renewcommand{\maketitle}{
\begin{center}
\vspace{2ex}
{\huge \textsc{\@title}}
\vspace{1ex}
\\
\linia\\
\@author \hfill \@date
\vspace{4ex}
\end{center}
}
\makeatother
%%%

% custom footers and headers
\usepackage{fancyhdr,lastpage}
\pagestyle{fancy}
\lhead{}
\chead{}
\rhead{}
\lfoot{Assignment 2}
\cfoot{}
\rfoot{Page \thepage\ /\ \pageref*{LastPage}}
\renewcommand{\headrulewidth}{0pt}
\renewcommand{\footrulewidth}{0pt}
%

%%%----------%%%----------%%%----------%%%----------%%%

\begin{document}

\title{EE604A - Digital Image Processing Assignment}

\author{Satya Prakash Panuganti, 14610}

\date{30 August, 2017}

\maketitle

\section*{Image Sources}
Sunset  (high.jpg)  : \url{https://www.flickr.com/photos/mediaflex/4190084346} \\
Carving (low.jpg)   : \url{https://www.flickr.com/photos/30440933@N06/2847993403} \\
Dog     (small.jpg) : \url{https://res.cloudinary.com/rover-com/image/upload/a_exif,c_fill,f_jpg,fl_progressive,g_face:center,h_100,q_80,w_100/remote/images/pets/4NpPzO8N/50e4a023d9/original.jpg}
\section*{Solution 1}
\subsection*{(a)}
Code has been written in two MATLAB\textregistered\ files :
\begin{itemize}
\item Q1\textbackslash lloyd\_max\_quantizer.m : The file contating the function
\item Q1\textbackslash Q1.m : The script to perform the required tasks.
\end{itemize}
\subsection*{(b)}
The 4 representation levels are :

The corresponding transition levels are :

\subsection*{(c)}
\subsection*{(d)}
\subsection*{(e)}

\section*{Solution 2}

A C++ function, EE604A::histogram\_matching(), for histogram matching of cv::Mat images has been developed.
The code required for the matching function is present in :
\begin{itemize}
\item Q2\textbackslash src\textbackslash histogram\_matching.cxx
\item Q2\textbackslash src\textbackslash histogram\_matching.h 
\end{itemize}
A small program which uses the histogram matching function is present in Q2\textbackslash src\textbackslash Q2.cxx

In order to run the code, the following steps need to be followed :
\begin{enumerate}
\item Enter Q2\textbackslash src.
\item run build.sh (OpenCV needs to installed and a version of g++ supporting c++14 needs to be present)
\item Execute Q2 (the binary file) with the reference and target relative/absolute image paths. Eg. [./Q2 ../../images/high.jpg ../../images/low.jpg] or [./Q2 ../../images/low.jpg ../../images/high.jpg] (On Ubuntu)
\item The program can be closed by Ctrl+C on the terminal or by pressing 'q' when one of the image windows is open.
\end{enumerate}
REMARK : A simple plot of the the histograms can be obtained by toggling 
\section*{Solution 3}

\begin{center}
%\includegraphics[scale = 0.37]{../images/q1output.jpg}
%The images obatained after projecting the LIDAR points onto them.
\end{center}

\section*{Solution 4}
$$\text{We are assuming that }\eta_i (x_1, y_1)\ and\ \eta_j (x_2, y_2)\text{ are independent if i }\neq j\ or\ x_1 \neq x_2\ or\ y_1 \neq y_w.$$
\begin{align}
\therefore\ if\ i \neq j,\ E[\eta_i(x, y)\eta_j(x, y)] & = 0
\end{align}
\begin{align*}
We\ have,\ g_i(x, y) & = f_i(x, y) + \eta_i(x, y) \\
Also,\ f_i(x, y) & = f(x, y) \\
Now,\ \hat{g}(x, y)& = \frac{1}{K} \Sigma_{i = 1}^K g_i(x, y) \\
& = \frac{1}{K}K f(x, y) + \frac{1}{K}\Sigma_{i = 1}^K \eta_i(x, y) \\
& = f(x, y) + \frac{1}{K}\Sigma_{i = 1}^K \eta_i(x, y) \\
Now,\ noise\ variance,\ E[(\hat{g(x, y)} - f(x, y))^2] & = E[(\frac{1}{K}\Sigma_{i = 1}^K \eta_i(x, y))^2] \\
& = \frac{1}{K^2}\Sigma_{i = 1}^K\Sigma_{j = 1}^K E[\eta_i(x, y)\eta_j(x, y)] \\
& = \frac{1}{K^2}\Sigma_{i = 1}^K E[\eta_i(x, y)^2] & from\ (1) \\
& = \frac{1}{K^2}\Sigma_{i = 1}^K \sigma^2 & \because\ E[\eta_i(x, y)^2] = \sigma^2 \\
& = \frac{\sigma^2}{K} \\
q.e.d
\end{align*}

\section*{Solution 5}

Let $f(x, y, z) : \Re^3 \rightarrow \Re$. Let $[u\ v\ w]^T$ be the position of $[x\ y\ z]^T$ in a rotated cooridnate frame.
The relationship between $[u\ v\ w]^T$ and $[x\ y\ z]^T$ is given by :
\begin{align}
\begin{bmatrix}
	u \\
	v \\
	w
\end{bmatrix} & =
	R
\begin{bmatrix}
	x \\
	y \\
	z
\end{bmatrix} \\
where,\ R & = [R_1\ R_2\ R_3] \\
& = \begin{bmatrix}
		R_{11} & R_{12} & R_{13} \\
		R_{21} & R_{22} & R_{23} \\
		R_{31} & R_{32} & R_{33}
	\end{bmatrix} \\
Known\ properties\ of\ R,\ |R_1| & = 1 \\
|R_2| & = 1 \\
R_1^T R_2 & = 0 \\
R_3  & = R_1 \times R_2 \\
\implies\ |R_3| & = 1 \\
R_1^T R_3 & = R_2^T R_3 = 0 \\
R^{-1} & = R^T \\
We\ can\ also\ write,\
\begin{bmatrix}
	x \\
	y \\
	z
\end{bmatrix} & =
	R^T
\begin{bmatrix}
	u \\
	v \\
	w
\end{bmatrix} & from\ (2),\ (9) \\
Now,\ \nabla^2 f(u, v, w) & = \frac{\partial^2 f}{\partial^2 u} +
							  \frac{\partial^2 f}{\partial^2 v} +
						      \frac{\partial^2 f}{\partial^2 w} \\
	\frac{\partial f}{\partial u} & = \frac{\partial f}{\partial x}\frac{\partial x}{\partial u} + \frac{\partial f}{\partial y}\frac{\partial y}{\partial u} + \frac{\partial f}{\partial z}\frac{\partial z}{\partial u} \\
& = R_{11}\frac{\partial f}{\partial x} +
	R_{12}\frac{\partial f}{\partial y} +
	R_{13}\frac{\partial f}{\partial z} & from\ (4), (10) \\
Similarily,\ \frac{\partial f}{\partial v}
& = R_{21}\frac{\partial f}{\partial x} +
	R_{22}\frac{\partial f}{\partial y} +
	R_{23}\frac{\partial f}{\partial z}\\
\ \frac{\partial f}{\partial w}
& = R_{31}\frac{\partial f}{\partial x} +
	R_{32}\frac{\partial f}{\partial y} +
	R_{33}\frac{\partial f}{\partial z}
\end{align}
\begin{align}
\therefore\ \frac{\partial^2 f}{\partial^2 u} & =
R_{11}(\frac{\partial^2 f}{\partial^2 x} \frac{\partial x}{\partial u} +
	  \frac{\partial^2 f}{\partial y \partial x} \frac{\partial y}{\partial u} +	  \frac{\partial^2 f}{\partial z \partial x} \frac{\partial z}{\partial u})\notag \\
& + R_{12}(\frac{\partial^2 f}{\partial x \partial y} \frac{\partial x}{\partial u} + \frac{\partial^2 f}{\partial^2 y} \frac{\partial y}{\partial u} +	 \frac{\partial^2 f}{\partial z \partial x} \frac{\partial z}{\partial u}) \notag \\
& + R_{13}(\frac{\partial^2 f}{\partial x \partial z} \frac{\partial x}{\partial u} + \frac{\partial^2 f}{\partial y \partial z} \frac{\partial y}{\partial u} +	 \frac{\partial^2 f}{\partial^2 z} \frac{\partial z}{\partial u}) & from\ (13)\ and\ chain\ rule \\
& = R_{11}^2\frac{\partial^2 f}{\partial^2 x} +
	R_{11}R_{12}\frac{\partial^2 f}{\partial y \partial x} +
	R_{11}R_{13}\frac{\partial^2 f}{\partial z \partial x}\notag \\
& + R_{12}R_{11}\frac{\partial^2 f}{\partial x \partial y} +
	R_{12}^2\frac{\partial^2 f}{\partial^2 y} +
	R_{12}R_{13}\frac{\partial^2 f}{\partial z \partial y}\notag \\
& + R_{13}R_{11}\frac{\partial^2 f}{\partial x \partial z} +
	R_{13}R_{12}\frac{\partial^2 f}{\partial y \partial z} +
	R_{13}^2\frac{\partial^2 f}{\partial^2 z} & from\ (4), (10) \\
Similarily,\ \frac{\partial^2 f}{\partial^2 v} & =
	R_{21}^2\frac{\partial^2 f}{\partial^2 x} +
	R_{21}R_{22}\frac{\partial^2 f}{\partial y \partial x} +
	R_{21}R_{23}\frac{\partial^2 f}{\partial z \partial x}\notag \\
& + R_{22}R_{21}\frac{\partial^2 f}{\partial x \partial y} +
	R_{22}^2\frac{\partial^2 f}{\partial^2 y} +
	R_{22}R_{23}\frac{\partial^2 f}{\partial z \partial y}\notag \\
& + R_{23}R_{21}\frac{\partial^2 f}{\partial x \partial z} +
	R_{23}R_{22}\frac{\partial^2 f}{\partial y \partial z} +
	R_{23}^2\frac{\partial^2 f}{\partial^2 z} \\
\frac{\partial^2 f}{\partial^2 w} & =
	R_{31}^2\frac{\partial^2 f}{\partial^2 x} +
	R_{31}R_{32}\frac{\partial^2 f}{\partial y \partial x} +
	R_{31}R_{33}\frac{\partial^2 f}{\partial z \partial x}\notag \\
& + R_{32}R_{31}\frac{\partial^2 f}{\partial x \partial y} +
	R_{32}^2\frac{\partial^2 f}{\partial^2 y} +
	R_{32}R_{33}\frac{\partial^2 f}{\partial z \partial y}\notag \\
& + R_{33}R_{31}\frac{\partial^2 f}{\partial x \partial z} +
	R_{33}R_{32}\frac{\partial^2 f}{\partial y \partial z} +
	R_{33}^2\frac{\partial^2 f}{\partial^2 z}
\end{align}
On performing $(17) + (18) + (19)$, we get using (5), (6), (7), (9) and (10)
\begin{align}
\frac{\partial^2 f}{\partial^2 u} +
\frac{\partial^2 f}{\partial^2 v} +
\frac{\partial^2 f}{\partial^2 w} & =
\frac{\partial^2 f}{\partial^2 u} +
\frac{\partial^2 f}{\partial^2 v} +
\frac{\partial^2 f}{\partial^2 w}\\
\implies\ \nabla^2 f(u, v, w) & = \frac{\partial^2 f}{\partial^2 u} +
							  \frac{\partial^2 f}{\partial^2 v} +
						      \frac{\partial^2 f}{\partial^2 w} \\
						      & = \frac{\partial^2 f}{\partial^2 x} +
							  \frac{\partial^2 f}{\partial^2 y} +
						      \frac{\partial^2 f}{\partial^2 z} & from\ (22) \\
						      & =  \nabla^2 f(x, y, z)
\end{align}
\begin{align*}
q.e.d
\end{align*}
\section*{Solution 6}
\end{document}